In 2001, Peter Loscocco and Stephen Smalley wrote an article introducing the \gls{sel} \cite{LS01}. The main reason that led to the development of such mechanism was the flawed assumption that adequate security should reside in applications, leaving the role of the operating system behind \cite{LSMTTF98}. They supported the idea that secure applications require secure operating systems. A strong concept related to operating systems security is \textit{access control policy}. In a simple manner, this term specifies which of the operations associated with an object are authorized to perform. Linux kernel inherited from the UNIX security model the \gls{dac} that allows the owner of an object to set the security policy for that object (the control of access is based on the discretion of the owner). However, this model of access control brings some advantages. For instance, every program executed by a certain user receives all of the privileges associated with that user. Therefore it is able to change the permissions of all user's objects, creating potential security threats. In this sense, a \gls{mac} was purposed to protect the system against vulnerabilities left by other access control models. In \gls{mac} the operating system constrains the ability of a subject to perform an operation on an object, depending on the security attributes. Whenever a subject attempts to access an object, an authorization rule enforced by the operating system kernel checks these security attributes in order to allow or deny the access.\\

\noindent
At the Linux Kernel 2.5 Summit, the \gls{nsa}, based on the security issues previously mentioned, presented their work on \gls{sel}, a security mechanism of a flexible access control architecture in the Linux kernel. \gls{nsa} reiterated the need for such support in the mainstream Linux kernel. Other projects were presented to enforce access policies, namely \gls{dte}, \gls{lids} and POSIX.1e capabilities. Given these projects, Linus Torvalds decided to provide a general framework for security policy, named \gls{lsm}. This framework allow many different access control models to be implemented as loadable kernel modules. Linus enforced that \gls{lsm} should be truly generic, where using a different security model was a question of loading a different kernel module. The framework should also be conceptually simple, minimally invasive and efficient. At last, the mechanism should be able to support the POSIX.1e capabilities logic as an optional security module \cite{WCSMK02}.\\

\noindent
This security framework has motivated developers and gave them freedom to build their own \gls{lsm} according to how they consider kernel objects should be accessed. \gls{sel}\footnote{http://selinuxproject.org} was originally made by the \gls{nsa} and has been in the mainstream kernel since version 2.6 (December 2003). \gls{sel} presents three forms of access control, \gls{dte}, \gls{rbac} and \gls{mls}. It uses the filesystem to mark executables when keeping track of permissions.\\

\noindent
Smack (Simple Mandatory Access Control Kernel)\footnote{http://schaufler-ca.com} has been in the mainstream kernel since version 2.6.26 (July 2008). This module was implement to provide simplicity to users. The complexity of \gls{dte} is avoided by defining access controls in terms of the access modes already in use.\\

\noindent
AppArmor (Application Armor)\footnote{http://wiki.apparmor.net} was originally developed by Immunix, which was a commercial operating system acquired by Novell in 2005. Novell laid off AppArmor programmers in 2007, but they continued the work. Since 2009, Canonical contributes to the project. This module has been in the mainstream Linux kernel since version 2.6.36 (October 2010). While \gls{sel} is based on applying labels to files, AppArmor uses pathnames to make security decisions. For instance, two different security policies may be applied  to the same file if that file is accessed by way of two different names. Many Linux administrators claim that AppArmor is the easiest security module to configure. Yet, others state that a pathname-based mechanism is insecure and that security policies should apply directly to objects (or to labels attached directly to objects) rather than to names given to objects.\\

\noindent
TOMOYO Linux\footnote{http://tomoyo.sourceforge.jp} is another \gls{mac} implementation for Linux. It has been in the mainstream kernel since version 2.6.30 (June 2009). This security mechanism follows the pathname-based philosophy, like AppArmor. TOMOYO Linux focuses on the behavior of a system, allowing each process to declare behaviors and resources needed to achieve its purpose. A precise comparison chart is available at \url{http://tomoyo.sourceforge.jp/wiki-e/?WhatIs#comparison}.\\

\noindent
Recently, Yama has been added to the mainstream kernel since version 3.4 (May 2012). Yama is a \gls{lsm} that collects a number of system-wide \gls{dac}
security protections that are not handled by the core kernel itself.\\

\noindent
Since the first release of the \gls{lsm} framework that new updates are commited in almost every new version of the Linux kernel. It is important to mention that between version 2.6.25 and 2.6.27, the \gls{lsm} boot engine changed and became no longer a removable module. Since then, the \gls{lsm} is loaded only when the kernel is compiled.