Os dispositivos móveis que correm o sistema operativo Android são amplamente usados para navegar na internet e aceder um vasto leque de serviços online. No entanto, o facto de ser uma plataforma móvel usada à escala global, coloca-a como um alvo extremamente apetecido para ataques maliciosos que tentam identificar e explorar potenciais vulnerabilidades, a fim de aceder a dados privados. Por outro lado, os dispositivos Android podem ser usados como uma valiosa ferramenta de auditoria móvel.

O objetivo final deste projeto passa pelo desenvolvimento de um conjunto de ferramentas que possibilitem aos dispositivos que suportam Android avaliar redes de internet em termos de segurança. Estas ferramentas deverão incluir filtragem de tráfego, mapeação de redes, avaliação de vulnerabilidades e deteção de intrusões. Este projecto pretende também alertar os utilizadores para os perigos do uso de internet em dispositivos Android.